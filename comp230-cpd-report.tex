% Please do not change the document class
\documentclass{scrartcl}

% Please do not change these packages
\usepackage[hidelinks]{hyperref}
\usepackage[none]{hyphenat}
\usepackage{setspace}
\usepackage{url}
\doublespace

\title{COMP230 - CPD Report}
\author{1706165}

\begin{document}

\maketitle

\section{Introduction}
In this update of my on-going Continual Personal Development, I will try to raise a few things that I need to work on as an individual to bolster my chances of becoming a well-rounded individual and hopefully securing a financially safe future. To tackle these issues I will be using S.M.A.R.T targets to help solidify what I need to do to assure these issues don't happen again. In this report, I will be covering Prioritization, Confidence, Communication, Public Speaking and finally, Research.


\section{Prioritization}
Having the ability to prioritize important tasks over medial and generally less time constraining tasks is crucial because if the complexity of the work is high, that task needs to be on the top of the to-do list instead of focusing on everything else surrounding it. As a result of bad prioritization, work will get neglected and put off until the deadline comes around and then the trouble starts. I found this to be an important aspect as I have experienced this with the Graphics/Simulation module that I have actively put-off because of its sheer complexity but instead, I should have focused on that task above everything else because I knew how difficult it would be to start with. If I cannot prioritize tasks within a professional setting, I am sure to fall behind in development of features within the games which in turn, leads to being inactive in the whole development process. This could be fixed by simple management of tasks i.e. having all the modules on a list, researching the modules to see which ones would be easier and then dedicating the time after this process has been done. It would help to schedule every task and deadline to assure that I am on track.


\section{Confidence/Communication}
In general, I struggle with self-confidence and although it may not seem like it to the open eye as I do my part to make sure that it is not as prevalent. Being confident in your abilities as a professional is the foundation for helping other people with their problems as well as my own. Especially with programming, If I am not confident in my programming abilities and I am expected to help someone, my solutions to this person's potential problems are very limited. Having the confidence to approach an issue with less self-judgement is the ideal goal because then I am only
looking for solutions instead of more problems. This issue also stems into the Communication issues
I have encountered because you need to have the confidence to be able to reach out for help, no matter how dire the situation is/could be. It is very easy to say that a task is impossible without
having the confidence to approach it with the right mindset which is something that I need to work
on as an individual. A potential fix to these issues would be being active in more social environments, generally speaking to people more would help solve some of the confidence issues. A new hobby such as the gym or a circle of friends with similar interests.   


\section{Public Speaking}
Having issues with self-esteem are making Public Speaking very hard as even in front of an small audience, I crumble pretty easily. The pressure of retaining attention from other peers whilst at the same time conducting words on the spot is challenging to say the least. A part of the issue is that I forget to relax and be myself instead of being incredibly anxious. A lot of the presentations that we had to do for the past two years were very much expected to be professional and as such, I don't feel like there's any room for a joke or a laugh because everybody is usually tense during these moments. The more preparation that is done before the performance adds to the angst before actually presenting and it does not feel as natural to me when I am up there. In this case, I think the poison is the solution and as such, presenting to a small group of people may help build toward more successful presentations in the process. It would also help if presentations were more of a consistent part of the module as a skill to develop over-time, reviewing these presentations would also serve to be useful as good presentations can serve as foundation for future presenting opportunities.


\section{Research}
This is something that I have been neglecting and have neglected for a while. My interest in researching high-level programming techniques and how I can implement them into my own code has
been very lack-luster because I just cannot find it interesting to save my life. However, if I want
to move on as a professional, I need to keep up with the latest programming techniques to make sure
that my way of doing this isn't outdated in the industry. In addition, being up-to-date with the latest tools within the popular engines would expand my own understanding of what can be achieved/what cannot as well as a having a way to help other people if the resource has proven to be helpful to me via. a recommendation. A way to raise that interest level is to some coding practice every single day so that when coding, I can discover faster ways of doing various operations to be more code efficient. The code practice could be geared towards my own interest and as such, I may find it more interesting instead of delving into OpenGL on the weekends and having no clue on how to advance the application. Building up a foundation of programming knowledge through a series of coding tasks getting gradually more difficult would assure that I am developing my coding knowledge.


\section{Conclusion}
In conclusion, I have a lot of work to do. This time however, the issues are slightly different which at least tells me that I am growing towards something which is exciting but at the same time, the same issues being Research and Public Speaking, I have not really improved on in any significant way which is something to be noted for the next CPD report. In short-term, I feel that Prioritization is very important as I have learned with the Graphics module and just general structure or a schedule could go a long way which is something that I learned within this study block. Managing my time effectively is something that I struggle with but with simple organization techniques, it can become easier. My previous goal of developing a "Super Mario Bros" standard game is proving to be unstable as a goal because it is just too broad and having that same sort of impact isn't something that is possible or even if it was, it wouldn't bring me joy. Need a new goal.


\end{document}
